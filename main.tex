\documentclass[12pt,a4paper]{article}

% Page setup
\usepackage[a4paper,top=1in,bottom=1in,left=0.5in,right=0.5in]{geometry}
\usepackage{setspace}
\onehalfspacing

\usepackage{etoolbox}
\usepackage[nonumberlist]{glossaries}
\usepackage{hyperref}
\usepackage{graphicx}
\usepackage{float}
\usepackage{xcolor}
\usepackage{colortbl}

% Table packages
\usepackage{array}
\usepackage{booktabs}
\usepackage{longtable}
\usepackage{tabularx}
\usepackage{multirow}
\usepackage{multicol}

% List formatting
\usepackage{enumitem}

% Thai language support
\usepackage{xunicode}
\usepackage{xltxtra}

% Optimized fonts for Thai language - using Medium weight
\setmainfont[
    Path=./font/Sarabun/,
    UprightFont=*-Regular,
    BoldFont=*-Medium,
    ItalicFont=*-Italic,
    BoldItalicFont=*-MediumItalic,
    Scale=1.0,
    Ligatures=TeX,
    WordSpace=1.2,
    PunctuationSpace=1.1
]{Sarabun}

\newfontfamily\thaifont[
    Path=./font/Sarabun/,
    UprightFont=*-Regular,
    BoldFont=*-Medium,
    ItalicFont=*-Italic,
    BoldItalicFont=*-MediumItalic,
    Scale=1.0,
    Ligatures=TeX,
    WordSpace=1.2,
    PunctuationSpace=1.1,
    Script=Thai,
    Language=Thai
]{Sarabun}

% Light weight font family optimized for Thai
\newfontfamily\thailightfont[
    Path=./font/Sarabun/,
    UprightFont=*-Light,
    BoldFont=*-Medium,
    ItalicFont=*-LightItalic,
    BoldItalicFont=*-MediumItalic,
    Scale=1.0,
    Ligatures=TeX,
    WordSpace=1.2,
    PunctuationSpace=1.1,
    Script=Thai,
    Language=Thai
]{Sarabun}

% Font commands for easy switching
\newcommand{\textlight}[1]{{\thailightfont #1}}
\newcommand{\thailight}{\thailightfont}

% Thai typography optimization commands
\newcommand{\optimalthai}[1]{{\thaifont #1}}

% Additional packages for better Thai support
\usepackage{polyglossia}
\setdefaultlanguage{thai}
\setotherlanguage{english}

% Thai line breaking settings - optimized
\XeTeXlinebreaklocale "th"
\XeTeXlinebreakskip = 0pt plus 2pt minus 1pt

% Thai typography enhancements
\frenchspacing 
\tolerance=1000
\emergencystretch=3em

% Thai punctuation spacing
\XeTeXinterchartokenstate = 1

% Better Thai paragraph spacing
\setlength{\parskip}{0.5\baselineskip plus 0.2\baselineskip minus 0.1\baselineskip}
\setlength{\parindent}{2em}

% Custom section numbering with period and reduced spacing
\renewcommand{\thesection}{\arabic{section}.}
\renewcommand{\thesubsection}{\arabic{section}.\arabic{subsection}.}
\renewcommand{\thesubsubsection}{\arabic{section}.\arabic{subsection}.\arabic{subsubsection}}

% Reduce space between section number and title with custom font weights
\usepackage{titlesec}
\titleformat{\section}
  {\normalfont\normalsize\fontseries{m}\selectfont}{\thesection}{0.3em}{}
\titlespacing*{\section}{0pt}{12pt plus 4pt minus 2pt}{6pt plus 2pt minus 2pt}
\titleformat{\subsection}
  {\normalfont\normalsize\fontseries{m}\selectfont}{\thesubsection}{0.3em}{}
\titlespacing*{\subsection}{0pt}{10pt plus 3pt minus 2pt}{4pt plus 2pt minus 1pt}
\titleformat{\subsubsection}
  {\normalfont\normalsize\fontseries{m}\selectfont}{\thesubsubsection}{0.3em}{}
\titlespacing*{\subsubsection}{0pt}{8pt plus 2pt minus 1pt}{3pt plus 1pt minus 1pt}

\begin{document}

\begin{center}
\hfill\textlight{ทก.01}\\[1cm]
\large\textbf{แบบเสนอโครงงานพิเศษ (ปริญญานิพนธ์)}\\[0.3cm]
\normalsize\textlight{สาขาวิชาวิศวกรรมสารสนเทศและเครือข่าย}\\[0.1cm]
\normalsize\textlight{ภาควิชาเทคโนโลยีสารสนเทศ}\\[0.1cm]
\normalsize\textlight{คณะเทคโนโลยีและการจัดการอุตสาหกรรม}\\[0.1cm]
\end{center}

\thispagestyle{empty}
\vspace{0.5cm}

% Thesis information
\section{ข้อมูลขั้นต้นของโครงงาน}
\begin{enumerate}[leftmargin=2cm]
\small
    \item[1.1] ชื่อโครงงาน
    \\ \textlight{(ภาษาไทย)} \hspace{0.5cm} {แพลตฟอร์มวิศวกรรมความรู้เชิงสัมพันธ์การปรากฏร่วม}
    \\ \textlight{(ภาษาอังกฤษ)} \hspace{0.04cm} {Co-Occurrence Knowledge Engineering Platform}

    \item[1.2] ชื่อนักศึกษาผู้ทำโครงงาน
    \\ \textlight{ชื่อ-นามสกุล} \hspace{0.4cm} {นายยงยุทธ ชวนขุนทด}
    \\ \textlight{สาขาวิชา} \hspace{0.935cm} {วิศวกรรมสารสนเทศและเครือข่าย}
    \\ \textlight{ภาควิชา} \hspace{1.12cm} {เทคโนโลยีสารสนเทศ}
    \\ \textlight{ภาคเรียนที่} \hspace{0.7cm} {1}
    \\ \textlight{ปีการศึกษา} \hspace{0.67cm} {2568}

    \item[1.3] ชื่ออาจารย์ที่ปรึกษา
    \\ \textlight{ชื่อ-นามสกุล} \hspace{0.47cm} {รองศาสตราจารย์ ดร. อนิราช มิ่งขวัญ}
\end{enumerate}

% Thesis details
\section{รายละเอียดของโครงงาน}
\begin{enumerate}[leftmargin=2cm]
\small
    \item[2.1] ความเป็นมาและความสำคัญของปัญหา
    \vspace{0.35cm}
    \\
    \textlight{
        \hspace{1cm}ในยุคดิจิทัลปัจจุบัน ข้อมูลและความรู้ที่เกิดขึ้นในแต่ละวันมีปริมาณที่เพิ่มขึ้นอย่างรวดเร็ว โดยเฉพาะอย่างยิ่งเอกสารทางวิชาการ หนังสือ และงานวิจัยต่าง ๆ ที่มีการเผยแพร่ในรูปแบบดิจิทัล เช่น ไฟล์ PDF ซึ่งเป็นแหล่งความรู้ที่มีคุณค่าสูง อย่างไรก็ตาม การจัดการ การวิเคราะห์ และการค้นหาความเชื่อมโยงระหว่างความรู้จากเอกสารเหล่านี้ยังคงเป็นปัญหาที่ท้าทาย

        \hspace{1cm}ปัญหาหลักที่พบในปัจจุบันคือ การที่ผู้ใช้งานไม่สามารถมองเห็นภาพรวมของความสัมพันธ์และความเชื่อมโยงระหว่างแนวคิดต่าง ๆ ที่ปรากฏในเอกสารหลายฉบับได้อย่างชัดเจน การอ่านและทำความเข้าใจเอกสารแต่ละฉบับแยกกันทำให้เกิดการสูญเสียโอกาสในการค้นพบความรู้ใหม่ที่อาจเกิดขึ้นจากการเชื่อมโยงข้อมูลจากแหล่งที่แตกต่างกัน

        \hspace{1cm}นอกจากนี้ การวิเคราะห์ความถี่ของการใช้คำและการปรากฏร่วมของความรู้ \textbf{(Co-occurrence)} ในเอกสารยังเป็นกระบวนการที่ซับซ้อนและใช้เวลามาก หากต้องทำด้วยมือหรือเครื่องมือพื้นฐาน ทำให้การสกัดความรู้และการสร้างความเข้าใจเชิงลึกจากเอกสารเป็นไปได้ยาก

        \hspace{1cm}ด้วยเหตุนี้ จึงจำเป็นต้องมีระบบที่สามารถแปลงเอกสารจากแหล่งต่าง ๆ ให้กลายเป็นกราฟเครือข่าย \textbf{(Network Graph)} ที่แสดงความสัมพันธ์และความเชื่อมโยงระหว่างแนวคิดได้อย่างชัดเจน รวมถึงสามารถสกัดส่วนของกราฟเพื่อนำไปผสมผสานกับข้อมูลจากแหล่งอื่น ๆ เพื่อสร้างความรู้ใหม่และค้นพบความเป็นไปได้ที่ไม่เคยมีมาก่อน ซึ่งจะช่วยเพิ่มประสิทธิภาพในการจัดการความรู้และส่งเสริมการเกิดนวัตกรรมใหม่ ๆ ในอนาคต
    }

    \item[2.2] วัตถุประสงค์ของโครงงาน
    \vspace{0.05cm}
    \textlight{
        \begin{enumerate}
            \item[2.2.1] เพื่อพัฒนาแพลตฟอร์มวิศวกรรมความรู้เชิงสัมพันธ์การปรากฏร่วมที่สามารถแปลงเอกสารให้กลายเป็นกราฟเครือข่ายความรู้ได้อย่างมีประสิทธิภาพ
            \item[2.2.2] เพื่อพัฒนาระบบวิเคราะห์การปรากฏร่วมของคำและแนวคิด (Co-occurrence Analysis) ที่สามารถระบุความถี่และความสัมพันธ์ระหว่างคำศัพท์ในเอกสารได้อย่างแม่นยำ
            \item[2.2.3] เพื่อสร้างส่วนติดต่อผู้ใช้ (User Interface) ที่ช่วยให้ผู้ใช้สามารถแสดงผลและโต้ตอบกับกราฟเครือข่ายความรู้ได้อย่างง่ายดายและเข้าใจง่าย
            \item[2.2.4] เพื่อพัฒนาฟีเจอร์การจัดการส่วนของกราฟ (Graph Management) เพื่อนำไปใช้ในการสร้างกราฟเครือข่ายใหม่หรือผสมผสานกับข้อมูลจากแหล่งอื่น
            \item[2.2.5] เพื่อพัฒนาระบบการผสมผสานความรู้จากหลายแหล่งข้อมูลเพื่อค้นหาความเชื่อมโยงและสร้างความรู้ใหม่ที่มีความเชื่อมโยงกัน
            \item[2.2.6] เพื่อพัฒนาระบบบูรณาการกับ Large Language Models (LLM) ที่สามารถใช้ข้อมูลจากกราฟเครือข่ายในการปรับปรุงความแม่นยำและประสิทธิภาพของการค้นหาและสกัดความรู้
        \end{enumerate}
    }
    
    \item[2.3] ขอบเขตของการทำโครงงานพิเศษ (Scope of Special Project)
    \vspace{0.05cm}
    \textlight{
        \begin{enumerate}
            \item[2.3.1] การพัฒนาระบบประมวลผลและวิเคราะห์เอกสาร เช่น PDF ที่สามารถดึงข้อความ วิเคราะห์โครงสร้าง และแยกแยะเนื้อหาสำคัญจากเอกสารได้อย่างมีประสิทธิภาพ รวมถึงการจัดการกับรูปแบบการจัดวางข้อความและภาษาที่หลากหลาย
            \item[2.3.2] การพัฒนาระบบวิเคราะห์การปรากฏร่วม (Co-occurrence Analysis) ที่สามารถ
            \begin{enumerate}
                \item[2.3.2.1] วิเคราะห์ความถี่ของคำและวลีในเอกสาร
                \item[2.3.2.2] ระบุความสัมพันธ์และการปรากฏร่วมของแนวคิดต่าง ๆ
                \item[2.3.2.3] คำนวณค่าความแข็งแกร่งของความเชื่อมโยงระหว่างคำหรือแนวคิด
                \item[2.3.2.4] สร้างเมทริกซ์ความสัมพันธ์สำหรับการสร้างกราฟเครือข่าย
            \end{enumerate}
            \item[2.3.3] การพัฒนาระบบสร้างและจัดการกราฟเครือข่ายความรู้ (Knowledge Network Graph) ที่สามารถ
            \begin{enumerate}
                \item[2.3.3.1] แปลงข้อมูลการวิเคราะห์ให้เป็นโครงสร้างกราฟ
                \item[2.3.3.2] จัดกลุ่มและจัดระเบียบโหนดและขอบเชื่อมตามความสัมพันธ์
                \item[2.3.3.3] คำนวณคุณสมบัติของกราฟ เช่น ความหนาแน่น ความเป็นศูนย์กลาง
                \item[2.3.3.4] สนับสนุนการแสดงผลแบบ Interactive และ Dynamic
            \end{enumerate}
            \item[2.3.4] การพัฒนาส่วนติดต่อผู้ใช้ (User Interface) ที่มีคุณสมบัติ
            \begin{enumerate}
                \item[2.3.4.1] อัปโหลดและจัดการไฟล์ PDF
                \item[2.3.4.2] แสดงผลกราฟเครือข่ายแบบโต้ตอบได้
                \item[2.3.4.3] เครื่องมือการค้นหาและกรองข้อมูล
                \item[2.3.4.4] ส่งออกผลลัพธ์ในรูปแบบต่าง ๆ (รูปภาพ, JSON, CSV)
            \end{enumerate}
            \item[2.3.5] การพัฒนาฟีเจอร์การสกัดและจัดการส่วนของกราฟ (Graph Management) ที่สามารถ
            \begin{enumerate}
                \item[2.3.5.1] เลือกและสกัดส่วนที่สนใจจากกราฟใหญ่
                \item[2.3.5.2] บันทึกและจัดเก็บส่วนกราฟที่สกัดไว้
                \item[2.3.5.3] ผสมผสานกราฟจากหลายแหล่งข้อมูล
                \item[2.3.5.4] สร้างกราฟใหม่จากการรวมข้อมูลหลายเอกสาร
            \end{enumerate}
            \item[2.3.6] การพัฒนาระบบฐานข้อมูลสำหรับจัดเก็บข้อมูล
            \begin{enumerate}
                \item[2.3.6.1] ข้อมูลเอกสารและเมตาดาต้า
                \item[2.3.6.2] ผลการวิเคราะห์และกราฟเครือข่าย
                \item[2.3.6.3] ประวัติการทำงานและการแก้ไข
                \item[2.3.6.4] การตั้งค่าและ Preferences ของผู้ใช้
            \end{enumerate}
            \item[2.3.7] การพัฒนาระบบบูรณาการกับ Large Language Models (LLM) ที่สามารถ
            \begin{enumerate}
                \item[2.3.7.1] แปลงข้อมูล Network Graph ให้เป็นรูปแบบที่ LLM สามารถเข้าใจและประมวลผลได้
                \item[2.3.7.2] สร้างระบบ Query Interface ที่ช่วยให้ผู้ใช้สามารถสอบถามข้อมูลจากกราฟผ่าน LLM ได้
                \item[2.3.7.3] พัฒนา Context-aware Search ที่ใช้ความรู้จากกราฟเครือข่ายในการปรับปรุงความแม่นยำของการค้นหา
                \item[2.3.7.4] สร้างระบบ Knowledge Discovery ที่ใช้ LLM ในการวิเคราะห์และสกัดความรู้ใหม่จากความสัมพันธ์ในกราฟ
            \end{enumerate}
        \end{enumerate}
    }

    \item[2.4] รายละเอียดของทฤษฎีที่ใช้ในการจัดทำปริญญานิพนธ์
    \vspace{0.05cm}
    \textlight{
        \begin{enumerate}
            \item[2.4.1] สมมติฐาน หรือ ข้อตกลงเบื้องต้นในการจัดทำโครงงานพิเศษ (Assumption of the Study)
            \vspace{0.05cm}
            \begin{enumerate}
                \item[2.4.1.1] เอกสารที่นำเข้าสู่ระบบจะอยู่ในรูปแบบ PDF เป็นต้น ฯลฯ ที่มีข้อความที่สามารถสกัดได้ (Text-extractable) และมีคุณภาพเพียงพอสำหรับการประมวลผลด้วยเทคนิค Optical Character Recognition (OCR) ในกรณีที่จำเป็น
                \item[2.4.1.2] เอกสารที่ใช้ในการวิเคราะห์จะเป็นเอกสารทางวิชาการ งานวิจัย หรือเอกสารที่มีโครงสร้างและเนื้อหาที่ชัดเจน โดยมีการใช้คำศัพท์และแนวคิดที่สามารถระบุและวิเคราะห์ความสัมพันธ์ได้
                \item[2.4.1.3] ระบบจะทำงานภายใต้สมมติฐานที่ว่าการปรากฏร่วมของคำหรือแนวคิดในระยะทางที่ใกล้กันภายในเอกสารแสดงถึงความสัมพันธ์หรือความเชื่อมโยงระหว่างแนวคิดเหล่านั้น
                \item[2.4.1.4] ผู้ใช้งานระบบจะมีความรู้พื้นฐานในการตีความและวิเคราะห์กราฟเครือข่าย รวมถึงสามารถระบุความสำคัญและความหมายของความสัมพันธ์ที่แสดงในกราฟได้
                \item[2.4.1.5] ระบบจะมีประสิทธิภาพในการประมวลผลเอกสารที่มีขนาดปานกลางถึงใหญ่ โดยสมมติว่าทรัพยากรคอมพิวเตอร์ที่ใช้งานมีความสามารถเพียงพอสำหรับการประมวลผลและการแสดงผลกราฟเครือข่ายแบบโต้ตอบได้
                \item[2.4.1.6] การบูรณาการกับ Large Language Models (LLM) จะช่วยปรับปรุงความแม่นยำในการระบุและจัดกลุ่มแนวคิด โดยสมมติว่า LLM จะสามารถเข้าใจบริบทและความหมายของข้อความในเอกสารได้อย่างถูกต้อง
                \item[2.4.1.7] ผลลัพธ์ที่ได้จากระบบจะมีความเชื่อถือได้และสามารถนำไปใช้ในการตัดสินใจหรือการวิจัยเพิ่มเติมได้ โดยระบบจะมีกลไกในการตรวจสอบและปรับปรุงความแม่นยำของผลการวิเคราะห์
                \item[2.4.1.8] การผสมผสานข้อมูลจากหลายแหล่งจะช่วยสร้างความรู้ใหม่ที่มีคุณค่า โดยสมมติว่าข้อมูลจากแหล่งต่าง ๆ จะมีความเข้ากันได้และสามารถรวมเข้าด้วยกันได้อย่างมีความหมาย
            \end{enumerate}

            \item[2.4.2] คำจำกัดความ (Key Word)
            \vspace{0.05cm}
            \begin{enumerate}
                \item[2.4.2.1] \textbf{การปรากฏร่วม (Co-occurrence)} หมายถึง การที่คำหรือแนวคิดสองคำหรือมากกว่าปรากฏในตำแหน่งที่ใกล้เคียงกันภายในเอกสารหรือข้อความ ซึ่งแสดงถึงความสัมพันธ์หรือความเชื่อมโยงระหว่างแนวคิดเหล่านั้น
                \item[2.4.2.2] \textbf{วิศวกรรมความรู้ (Knowledge Engineering)} หมายถึง กระบวนการในการสกัด จัดระเบียบ จัดเก็บ และจัดการความรู้จากแหล่งข้อมูลต่าง ๆ เพื่อให้สามารถนำไปใช้งานได้อย่างมีประสิทธิภาพ
                \item[2.4.2.3] \textbf{แพลตฟอร์ม (Platform)} หมายถึง ระบบซอฟต์แวร์ที่ให้บริการและเครื่องมือครบครันสำหรับการดำเนินงานเฉพาะด้าน ในที่นี้คือการวิเคราะห์และจัดการความรู้เชิงสัมพันธ์
                \item[2.4.2.4] \textbf{กราฟเครือข่าย (Network Graph)} หมายถึง โครงสร้างข้อมูลที่ประกอบด้วยโหนด (Nodes) และขอบเชื่อม (Edges) ที่แสดงความสัมพันธ์ระหว่างแนวคิดหรือเอนทิตีต่าง ๆ ในรูปแบบที่เข้าใจได้ง่าย
                \item[2.4.2.5] \textbf{การวิเคราะห์การปรากฏร่วม (Co-occurrence Analysis)} หมายถึง เทคนิคการวิเคราะห์ข้อมูลที่ใช้ในการระบุและวัดความถี่ของการปรากฏร่วมของคำหรือแนวคิดในข้อความ เพื่อค้นหาความสัมพันธ์และรูปแบบต่าง ๆ
                \item[2.4.2.6] \textbf{โหนด (Node)} หมายถึง จุดหรือองค์ประกอบพื้นฐานในกราฟเครือข่ายที่แทนคำ แนวคิด หรือเอนทิตีต่าง ๆ ที่ได้จากการวิเคราะห์เอกสาร
                \item[2.4.2.7] \textbf{ขอบเชื่อม (Edge)} หมายถึง เส้นเชื่อมระหว่างโหนดในกราฟเครือข่ายที่แสดงความสัมพันธ์หรือความเชื่อมโยงระหว่างแนวคิดหรือเอนทิตีต่าง ๆ รวมถึงค่าน้ำหนักที่บ่งบอกถึงความแข็งแกร่งของความสัมพันธ์
                \item[2.4.2.8] \textbf{การสกัดข้อความ (Text Extraction)} หมายถึง กระบวนการในการดึงข้อความจากเอกสารดิจิทัล เช่น ไฟล์ PDF โดยใช้เทคนิคการประมวลผลเอกสารหรือ OCR (Optical Character Recognition)
                \item[2.4.2.9] \textbf{เมทริกซ์ความสัมพันธ์ (Relationship Matrix)} หมายถึง ตารางสองมิติที่แสดงค่าความแข็งแกร่งของความสัมพันธ์ระหว่างคำหรือแนวคิดต่าง ๆ ที่ได้จากการวิเคราะห์การปรากฏร่วม
                \item[2.4.2.10] \textbf{Large Language Models (LLM)} หมายถึง โมเดลปัญญาประดิษฐ์ที่ได้รับการฝึกฝนด้วยข้อมูลข้อความขนาดใหญ่ เพื่อให้สามารถเข้าใจและประมวลผลภาษาธรรมชาติได้อย่างมีประสิทธิภาพ
                \item[2.4.2.11] \textbf{การแสดงผลแบบโต้ตอบ (Interactive Visualization)} หมายถึง การแสดงผลข้อมูลในรูปแบบที่ผู้ใช้สามารถโต้ตอบและปรับเปลี่ยนมุมมองหรือรายละเอียดได้ตามต้องการ
                \item[2.4.2.12] \textbf{การจัดการส่วนของกราฟ (Graph Management)} หมายถึง ระบบที่ช่วยในการเลือก สกัด บันทึก และจัดการส่วนต่าง ๆ ของกราฟเครือข่ายเพื่อนำไปใช้งานหรือวิเคราะห์เพิ่มเติม
                \item[2.4.2.13] \textbf{การค้นพบความรู้ (Knowledge Discovery)} หมายถึง กระบวนการในการค้นหาและระบุรูปแบบ ความสัมพันธ์ หรือความรู้ใหม่ที่ซ่อนอยู่ในข้อมูลขนาดใหญ่ผ่านเทคนิคการวิเคราะห์ข้อมูลต่าง ๆ
            \end{enumerate}

            \vspace{3.7cm}

            \item[2.4.3] รายงานการค้นคว้า การศึกษา หรือการวิจัยที่เกี่ยวข้อง
            \vspace{0.05cm}
             \begin{enumerate}
                \item[2.4.3.1] \textbf{Centroid Terms as Text Representatives}
                \\
                งานวิจัยของ Mario M. Kubek, Herwig Unger (2016) เรื่อง "Centroid Terms as Text Representatives" ได้ศึกษาเทคนิคการสร้างกราฟความรู้จากข้อความ โดยเฉพาะการใช้เทคนิค Co-occurrence Analysis ในการระบุความสัมพันธ์ระหว่างเอนทิตีต่าง ๆ การศึกษานี้ได้ชี้ให้เห็นว่าอัลกอริทึมสำหรับการจัดกลุ่มและการจำแนกข้อความตามหัวข้อนั้นอาศัยข้อมูลเกี่ยวกับระยะทางและความคล้ายคลึงเชิงความหมาย วิธีการมานตรฐานที่อิงตามแบบจำลอง bag-of-words ในการกำหนดข้อมูลนี้จะให้เพียงการประมาณแบบคร่าว ๆ เกี่ยวกับความเกี่ยวข้องของข้อความ นอกจากนี้ วิธีการเหล่านี้ยังไม่สามารถค้นหาคำศัพท์ที่เป็นนามธรรมหรือคำที่สามารถอธิบายเนื้อหาของข้อความได้อย่างครอบคลุม งานวิจัยนี้จึงได้นำเสนอวิธีการใหม่ในการกำหนดคำศูนย์กลาง (Centroid Terms) ในข้อความและการประเมินความคล้ายคลึงโดยใช้คำที่เป็นตัวแทนเหล่านั้น ซึ่งแสดงให้เห็นว่าการวิเคราะห์การปรากฏร่วมสามารถช่วยในการค้นพบรูปแบบความสัมพันธ์ที่ซ่อนอยู่ในข้อมูลขนาดใหญ่ได้อย่างมีประสิทธิภาพ และสามารถนำไปประยุกต์ใช้ในการพัฒนาระบบวิศวกรรมความรู้เชิงสัมพันธ์ได้                 
                    \item[2.4.3.2] \textbf{Spreading activation: a fast calculation method for text centroids}
                \\
                งานวิจัยของ Mario M. Kubek, Thomas Böhme, Herwig Unger (2017) เรื่อง "Spreading activation: a fast calculation method for text centroids" ได้นำเสนอเทคนิคการคำนวณ Centroid Terms แบบใหม่ที่มีประสิทธิภาพสูง โดยใช้ Spreading Activation Algorithm ซึ่งเป็นเทคนิคที่ทำงานบนพื้นฐานของกราฟและหลักการทำงานแบบเฉพาะที่ (Local Working Principle) การศึกษานี้ชี้ให้เห็นว่า Centroids เป็นเครื่องมือที่สะดวกในการแสดงคำค้นหาและข้อความทั้งหมดด้วยคำศัพท์เชิงบรรยายเพียงคำเดียว ซึ่งสามารถนำไปใช้ในการกำหนดความคล้ายคลึงของเนื้อหาข้อความและการจัดกลุ่มเอกสารแบบลำดับชั้น อย่างไรก็ตาม การคำนวณตามคำจำกัดความแบบดั้งเดิมอาจใช้เวลามากและเป็นอุปสรรคต่อการนำไปใช้งานจริง ดังนั้น การพัฒนาอัลกอริทึมแบบใหม่ที่อิงตามกราฟ Co-occurrence จึงมีความสำคัญต่อการเพิ่มประสิทธิภาพการประมวลผล ซึ่งสอดคล้องกับแนวทางที่จะใช้ในโครงงานนี้สำหรับการวิเคราะห์การปรากฏร่วมและการสร้างกราฟเครือข่ายความรู้ที่มีประสิทธิภาพในการประมวลผลเอกสารขนาดใหญ่

                \item[2.4.3.3] \textbf{Enhancing Retrieval-Augmented Generation Systems by Text-Representing Centroid}
                \\
                การศึกษาของ Yanakorn Ruamsuk, Anirach Mingkhawn, Herwig Unger (2025) เรื่อง "Enhancing Retrieval-Augmented Generation Systems by Text-Representing Centroid" ได้นำเสนอแนวทางใหม่ในการปรับปรุงระบบ Retrieval-Augmented Generation (RAG) โดยการบูรณาการเทคนิค Text-Representing Centroid (TRC) เพื่อแก้ไขข้อจำกัดของฐานข้อมูลเวกเตอร์แบบดั้งเดิม วิธีการนี้สามารถรักษาความสัมพันธ์เชิงโครงสร้างและปรับตัวตามความซับซ้อนของเนื้อหา ส่งผลให้การค้นคืนข้อมูลมีประสิทธิภาพและความแม่นยำที่สูงขึ้น การมีส่วนสนับสนุนที่สำคัญได้แก่ การสร้างกราฟขั้นสูง อัลกอริทึมการให้คะแนนความเกี่ยวข้อง และการตรวจสอบความถูกต้องอย่างครอบคลุม พร้อมการอภิปรายเกี่ยวกับการประยุกต์ใช้ที่เป็นไปได้และการวิจัยในอนาคต หลักฐานเชิงประจักษ์แสดงให้เห็นว่าเทคนิค TRC สามารถบรรลุอัตราความสำเร็จ 75 เปอร์เซ็นต์จากคำถาม 100 ข้อ ซึ่งมีประสิทธิภาพเหนือกว่าวิธีการเวกเตอร์แบบดั้งเดิม การศึกษานี้มีความเกี่ยวข้องโดยตรงกับโครงงานที่เสนอ เนื่องจากแสดงให้เห็นถึงความเป็นไปได้ในการใช้เทคนิค Co-occurrence Analysis และ Centroid-based Methods ในการพัฒนาระบบวิศวกรรมความรู้ที่มีประสิทธิภาพสูง
            \end{enumerate}

            \vspace{2.3cm}

            \item[2.4.4] เนื้อหา เหตุผล และทฤษฎีที่สำคัญ
            \vspace{0.05cm}
            \\
            \hspace{1cm}โครงงาน Co-Occurrence Knowledge Engineering Platform นี้มีพื้นฐานทางทฤษฎีที่แข็งแกร่งและเหตุผลที่ชัดเจนในการพัฒนา โดยอาศัยหลักการทางวิศวกรรมความรู้และเทคนิคการวิเคราะห์ข้อมูลขั้นสูงหลายแนวทาง

            \textbf{เหตุผลในการพัฒนาโครงงาน}

            \hspace{1cm}ในยุคสารสนเทศปัจจุบัน ปริมาณข้อมูลและเอกสารดิจิทัลเพิ่มขึ้นอย่างรวดเร็ว แต่การจัดการและการค้นหาความเชื่อมโยงระหว่างความรู้จากแหล่งต่าง ๆ ยังคงเป็นความท้าทายสำคัญ ผู้ใช้งานส่วนใหญ่ไม่สามารถมองเห็นภาพรวมของความสัมพันธ์ระหว่างแนวคิดที่ปรากฏในเอกสารหลายฉบับได้อย่างชัดเจน การวิเคราะห์การปรากฏร่วม (Co-occurrence Analysis) แบบดั้งเดิมใช้เวลามากและซับซ้อน ดังนั้น จึงจำเป็นต้องมีระบบที่สามารถแปลงเอกสารให้เป็นกราฟเครือข่ายความรู้ที่เข้าใจได้ง่าย และสามารถผสมผสานข้อมูลจากหลายแหล่งเพื่อสร้างความรู้ใหม่

            \textbf{ทฤษฎีพื้นฐานที่ใช้ในการพัฒนา}

            \textbf{1. ทฤษฎีการปรากฏร่วม (Co-occurrence Theory)}

            \hspace{1cm}หลักการพื้นฐานของการวิเคราะห์การปรากฏร่วมอิงตามสมมติฐานที่ว่า คำหรือแนวคิดที่ปรากฏใกล้กันในข้อความมักจะมีความสัมพันธ์หรือความเชื่อมโยงกัน ทฤษฎีนี้ได้รับการสนับสนุนจากงานวิจัยของ Mario M. Kubek et al. (2016, 2017) ที่แสดงให้เห็นว่าการวิเคราะห์ Centroid Terms และ Spreading Activation Algorithm สามารถช่วยในการระบุความสัมพันธ์เชิงความหมายได้อย่างมีประสิทธิภาพ

            \textbf{2. ทฤษฎีกราฟและเครือข่าย (Graph Theory and Network Theory)}

            \hspace{1cm}การแสดงความรู้ในรูปแบบกราฟเครือข่ายอิงตามทฤษฎีกราฟ ซึ่งประกอบด้วยโหนด (Nodes) แทนแนวคิดหรือเอนทิตี และขอบเชื่อม (Edges) แทนความสัมพันธ์ ทฤษฎีนี้ช่วยให้สามารถคำนวณคุณสมบัติต่าง ๆ ของเครือข่าย เช่น ความหนาแน่น (Density) ความเป็นศูนย์กลาง (Centrality) และการจัดกลุ่ม (Clustering) ซึ่งมีความสำคัญต่อการวิเคราะห์และการค้นพบความรู้

            \textbf{3. ทฤษฎีการประมวลผลภาษาธรรมชาติ (Natural Language Processing Theory)}

            \hspace{1cm}การสกัดข้อความและการวิเคราะห์เนื้อหาจากเอกสาร PDF อาศัยหลักการของ NLP รวมถึงเทคนิค Tokenization, Part-of-Speech Tagging, Named Entity Recognition และ Semantic Analysis เพื่อให้สามารถระบุและแยกแยะแนวคิดสำคัญได้อย่างแม่นยำ

            \textbf{4. ทฤษฎีการเรียนรู้ของเครื่อง (Machine Learning และ Deep Learning Theory)}

            \hspace{1cm}การบูรณาการกับ Large Language Models (LLM) อาศัยหลักการของ Deep Learning และ Transformer Architecture เพื่อปรับปรุงความแม่นยำในการระบุความสัมพันธ์และการทำ Semantic Reasoning งานวิจัยของ Yanakorn Ruamsuk et al. (2025) แสดงให้เห็นว่าการใช้ Text-Representing Centroid ร่วมกับ LLM สามารถบรรลุอัตราความสำเร็จ 75 เปอร์เซ็นต์ในการตอบคำถาม

            \textbf{5. ทฤษฎีการจัดการฐานข้อมูล (Database Management Theory)}

            \hspace{1cm}การจัดเก็บและจัดการข้อมูลกราฟอาศัยหลักการของ Graph Database และ NoSQL Database เพื่อรองรับการประมวลผลข้อมูลเชิงสัมพันธ์ที่ซับซ้อนและการ Query แบบ Real-time
        \end{enumerate}
    }

    \vspace{2.3cm}

    \item[2.5] วิธีดำเนินการจัดทำโครงงานพิเศษ
    \vspace{0.05cm}
    \\
    \textlight{
        \hspace{1cm}การพัฒนาแพลตฟอร์ม Co-Occurrence Knowledge Engineering Platform จะดำเนินการโดยใช้แนวทางการพัฒนาระบบแบบครบวงจร (Full-Stack Development) ร่วมกับเทคโนโลยีคลาวด์และเครื่องมือออกแบบสมัยใหม่ เพื่อให้ได้ระบบที่มีประสิทธิภาพ ปลอดภัย และใช้งานง่าย

        \vspace{0.5cm}

        \textbf{วิธีการดำเนินการหลัก}

        \begin{enumerate}
            \item[2.5.1] \textbf{การพัฒนาระบบบนคลาวด์เซิร์ฟเวอร์}
            \begin{enumerate}
                \item[2.5.1.1] ใช้คลาวด์เซิร์ฟเวอร์เป็นสภาพแวดล้อมหลักในการ Host โครงงาน
                \item[2.5.1.2] ติดตั้งและกำหนดค่าระบบปฏิบัติการ Linux (Server) สำหรับการประมวลผล
                \item[2.5.1.3] ปรับแต่งสภาพแวดล้อมสำหรับการพัฒนา TypeScript, Go และฐานข้อมูล
            \end{enumerate}

            \item[2.5.2] \textbf{การออกแบบ User Interface และ User Experience}
            \begin{enumerate}
                \item[2.5.2.1] ใช้เครื่องมือ Figma ในการออกแบบ Wireframes และ Mockups ทั้งหมด
                \item[2.5.2.2] สร้าง Design System ที่สอดคล้องกับการใช้งานของ Knowledge Engineering Platform
                \item[2.5.2.3] ออกแบบ Interactive Prototypes สำหรับการแสดงผลกราฟเครือข่าย
                \item[2.5.2.4] ทดสอบ Usability และปรับปรุงการออกแบบตามผลการทดสอบ
                \item[2.5.2.5] สร้าง Responsive Design เพื่อรองรับการใช้งานบนอุปกรณ์ต่าง ๆ
            \end{enumerate}

            \item[2.5.3] \textbf{การพัฒนาระบบจำลองและการทดสอบ}
            \begin{enumerate}
                \item[2.5.3.1] สร้างระบบจำลองการวิเคราะห์ Co-occurrence ด้วยข้อมูลตัวอย่าง
                \item[2.5.3.2] พัฒนา Proof of Concept สำหรับการสร้างกราฟเครือข่ายจากเอกสาร PDF
                \item[2.5.3.3] สร้างและทดสอบการบูรณาการกับ Large Language Models (LLM)
            \end{enumerate}

            \item[2.5.4] \textbf{การสร้างเอกสารและข้อกำหนดระบบ}
            \begin{enumerate}
                \item[2.5.4.1] จัดทำเอกสาร System Requirements และ Functional Specifications
                \item[2.5.4.2] สร้าง API Documentation สำหรับการใช้งานระบบ
                \item[2.5.4.3] เขียน User Manual และ Administrator Guide
                \item[2.5.4.4] จัดทำเอกสาร Technical Architecture และ Database Schema
                \item[2.5.4.5] สร้าง Test Cases และ Test Plans สำหรับการทดสอบระบบ
            \end{enumerate}

            \item[2.5.5] \textbf{การสร้างสภาพแวดล้อมความปลอดภัย}
            \begin{enumerate}
                \item[2.5.5.1] ติดตั้งและกำหนดค่า VPN Server สำหรับการเชื่อมต่อที่ปลอดภัย
                \item[2.5.5.2] สร้าง Private Network สำหรับการเข้าถึงฐานข้อมูลและทรัพยากรภายใน
                \item[2.5.5.3] ปรับแต่งระบบ Firewall และ Security Groups สำหรับการควบคุมการเข้าถึง
                \item[2.5.5.4] ใช้ SSL/TLS Certificates สำหรับการเข้ารหัสข้อมูล
            \end{enumerate}

            \item[2.5.6] \textbf{การพัฒนาและการทดสอบระบบ}
            \begin{enumerate}
                \item[2.5.6.1] สร้าง CI/CD Pipeline สำหรับการ Deploy และ Testing อัตโนมัติ
                \item[2.5.6.2] ทดสอบระบบด้วย Unit Testing, Integration Testing และ End-to-End Testing
                \item[2.5.6.3] ประเมินประสิทธิภาพระบบด้วย Performance Testing และ Load Testing
                \item[2.5.6.4] ทดสอบความปลอดภัยด้วย Security Testing และ Penetration Testing
            \end{enumerate}
        \end{enumerate}

        \textbf{สถานที่ดำเนินการ}

        \begin{enumerate}
            \item[2.5.7] \textbf{สถานที่หลักในการพัฒนา}
            \begin{enumerate}
                \item[2.5.7.1] ห้องปฏิบัติการคอมพิวเตอร์ ภาควิชาเทคโนโลยีสารสนเทศ คณะเทคโนโลยีและการจัดการอุตสาหกรรม สำหรับการพัฒนาและทดสอบระบบ
                \item[2.5.7.2] ห้องส่วนตัวที่มีการเชื่อมต่ออินเทอร์เน็ตความเร็วสูง สำหรับการพัฒนาและการออกแบบ
                \item[2.5.7.3] คลาวด์เซิร์ฟเวอร์ สำหรับการ Host และ Deploy ระบบ
            \end{enumerate}

            \item[2.5.8] \textbf{สภาพแวดล้อมการทำงาน}
            \begin{enumerate}
                \item[2.5.8.1] ระบบพัฒนาบนเครื่อง Local Development Environment (macOS/Linux)
                \item[2.5.8.2] ระบบทดสอบบน Staging Environment ในคลาวด์เซิร์ฟเวอร์
                \item[2.5.8.3] ระบบจริงบน Production Environment ที่มีการรักษาความปลอดภัยสูง
                \item[2.5.8.4] ระบบการออกแบบบน Figma Cloud Platform
            \end{enumerate}
        \end{enumerate}
    }

    \item[2.6] แผนกิจกรรมและตารางเวลาในการจัดทำ
    \item[2.6.1] แผนกิจกรรมหลักและระยะเวลา
    \begin{table}[htbp]
        \centering
        \caption{แผนการดำเนินงานภาคการศึกษาที่ 1}
        \vspace{0.4cm}
        \renewcommand{\arraystretch}{1.5}
        \footnotesize
        \begin{tabular}{|>{\arraybackslash}p{5.5cm}|>{\centering\arraybackslash}p{0.32cm}|>{\centering\arraybackslash}p{0.32cm}|>{\centering\arraybackslash}p{0.32cm}|>{\centering\arraybackslash}p{0.32cm}|>{\centering\arraybackslash}p{0.32cm}|>{\centering\arraybackslash}p{0.32cm}|>{\centering\arraybackslash}p{0.32cm}|>{\centering\arraybackslash}p{0.32cm}|>{\centering\arraybackslash}p{0.32cm}|>{\centering\arraybackslash}p{0.32cm}|>{\centering\arraybackslash}p{0.32cm}|>{\centering\arraybackslash}p{0.32cm}|>{\centering\arraybackslash}p{0.32cm}|>{\centering\arraybackslash}p{0.32cm}|>{\centering\arraybackslash}p{0.32cm}|>{\centering\arraybackslash}p{0.32cm}|}
            \hline
            \multirow{2}{*}{\textbf{ขั้นตอนการดำเนินการ}} & \multicolumn{4}{c|}{\textbf{ก.ค.}} & \multicolumn{4}{c|}{\textbf{ส.ค.}} & \multicolumn{4}{c|}{\textbf{ก.ย.}} & \multicolumn{4}{c|}{\textbf{ต.ค.}} \\
            \cline{2-17}
            & 1 & 2 & 3 & 4 & 1 & 2 & 3 & 4 & 1 & 2 & 3 & 4 & 1 & 2 & 3 & 4 \\
            \hline
            \textlight{1) วางแผนการพัฒนา} & \cellcolor{green!30} & & & & & & & & & & & & & & & \\
            \hline
            \textlight{2) เก็บ Requirement} & & \cellcolor{green!30} & & & & & & & & & & & & & & \\
            \hline
            \textlight{3) ออกแบบ UI และ UX} & & & \cellcolor{green!30} & \cellcolor{green!30} & \cellcolor{green!30} & & & & & & & & & & & \\
            \hline
            \textlight{4) ออกแบบสถาปัตยกรรมซอฟต์แวร์} & & & & & & \cellcolor{green!30} & \cellcolor{green!30} & \cellcolor{green!30} & & & & & & & & \\
            \hline
            \textlight{5) พัฒนาระบบส่วนแกนหลัก} & & & & & & & & & \cellcolor{green!30} & \cellcolor{green!30} & \cellcolor{green!30} & \cellcolor{green!30} & \cellcolor{green!30} & \cellcolor{green!30} & \cellcolor{green!30} & \cellcolor{green!30} \\
            \hline
        \end{tabular}
        \renewcommand{\arraystretch}{1}
    \end{table}

    \begin{table}[htbp]
        \centering
        \caption{แผนการดำเนินงานภาคการศึกษาที่ 2}
        \vspace{0.2cm}
        \renewcommand{\arraystretch}{1.5}
        \footnotesize
        \begin{tabular}{|>{\arraybackslash}p{5.5cm}|>{\centering\arraybackslash}p{0.32cm}|>{\centering\arraybackslash}p{0.32cm}|>{\centering\arraybackslash}p{0.32cm}|>{\centering\arraybackslash}p{0.32cm}|>{\centering\arraybackslash}p{0.32cm}|>{\centering\arraybackslash}p{0.32cm}|>{\centering\arraybackslash}p{0.32cm}|>{\centering\arraybackslash}p{0.32cm}|>{\centering\arraybackslash}p{0.32cm}|>{\centering\arraybackslash}p{0.32cm}|>{\centering\arraybackslash}p{0.32cm}|>{\centering\arraybackslash}p{0.32cm}|>{\centering\arraybackslash}p{0.32cm}|>{\centering\arraybackslash}p{0.32cm}|>{\centering\arraybackslash}p{0.32cm}|>{\centering\arraybackslash}p{0.32cm}|}
            \hline
            \multirow{2}{*}{\textbf{ขั้นตอนการดำเนินการ}} & \multicolumn{4}{c|}{\textbf{ธ.ค.}} & \multicolumn{4}{c|}{\textbf{ม.ค.}} & \multicolumn{4}{c|}{\textbf{ก.พ.}} & \multicolumn{4}{c|}{\textbf{มี.ค.}} \\
            \cline{2-17}
            & 1 & 2 & 3 & 4 & 1 & 2 & 3 & 4 & 1 & 2 & 3 & 4 & 1 & 2 & 3 & 4 \\
            \hline
            \textlight{6) ทดสอบระบบส่วนแกนหลัก} & \cellcolor{green!30} & \cellcolor{green!30} & \cellcolor{green!30} & \cellcolor{green!30} & & & & & & & & & & & & \\
            \hline
            \textlight{7) ทดสอบความปลอดภัยของระบบ} & & & & & \cellcolor{green!30} & \cellcolor{green!30} & \cellcolor{green!30} & \cellcolor{green!30} & & & & & & & & \\
            \hline
            \textlight{8) ทดสอบสภาพแวดล้อมจริง} & & & & & & & & & \cellcolor{green!30} & \cellcolor{green!30} & & & & & & \\
            \hline
            \textlight{9) จัดทำเอกสารแพลตฟอร์ม} & & & & & & & & & & & \cellcolor{green!30} & \cellcolor{green!30} & \cellcolor{green!30} & \cellcolor{green!30} & & \\
            \hline
            \textlight{10) ส่งมอบแพลตฟอร์ม} & & & & & & & & & & & & & & & \cellcolor{green!30} & \cellcolor{green!30} \\
            \hline
        \end{tabular}
        \renewcommand{\arraystretch}{1}
    \end{table}

    \vspace{0.4cm}

    \item[2.6.2] แผนภูมิขั้นตอนการจัดทำโครงงานพิเศษ โดยละเอียด
    \vspace{0.05cm}
    \\
    \textlight{
        \hspace{1cm}การพัฒนาแพลตฟอร์ม Co-Occurrence Knowledge Engineering Platform จะดำเนินการตามผังงานที่มีการแบ่งขั้นตอนและกิจกรรมอย่างชัดเจน พร้อมทั้งกำหนดอัตราส่วนของแต่ละกิจกรรมเป็นร้อยละเพื่อการติดตามความก้าวหน้าอย่างมีประสิทธิภาพ

        \vspace{0.5cm}

        \textbf{ผังงานการดำเนินงาน (Development Flow Chart)}

        \vspace{0.3cm}

        \textbf{Phase 1: การวางแผนและวิเคราะห์ (Planning \& Analysis Phase)} \textlight{- 15\% ของโครงงานทั้งหมด}

        \begin{enumerate}
            \item[1.1] \textbf{การวางแผนการพัฒนา (5\%)} 
            \begin{enumerate}
                \item[1.1.1] กำหนดขอบเขตโครงงานและวัตถุประสงค์
                \item[1.1.2] วิเคราะห์ความเป็นไปได้ทางเทคนิค (Technical Feasibility)
                \item[1.1.3] จัดทำ Project Charter และ Timeline
                \item[1.1.4] กำหนดทรัพยากรและงบประมาณ
            \end{enumerate}
            
            \item[1.2] \textbf{การเก็บความต้องการ (Requirements Gathering) (10\%)}
            \begin{enumerate}
                \item[1.2.1] วิเคราะห์ Functional Requirements
                \item[1.2.2] วิเคราะห์ Non-Functional Requirements
                \item[1.2.3] สร้าง Use Case Diagrams และ User Stories
                \item[1.2.4] กำหนด Acceptance Criteria สำหรับแต่ละฟีเจอร์
                \item[1.2.5] วิเคราะห์ข้อจำกัดและความเสี่ยง
            \end{enumerate}
        \end{enumerate}

        \textbf{Phase 2: การออกแบบระบบ (System Design Phase)} \textlight{- 25\% ของโครงงานทั้งหมด}

        \begin{enumerate}
            \item[2.1] \textbf{การออกแบบ UI และ UX (12\%)}
            \begin{enumerate}
                \item[2.1.1] สร้าง User Persona และ User Journey Mapping
                \item[2.1.2] ออกแบบ Wireframes และ Mockups ด้วย Figma
                \item[2.1.3] สร้าง Interactive Prototypes
                \item[2.1.4] ทดสอบ Usability และปรับปรุงการออกแบบ
                \item[2.1.5] สร้าง Design System และ Component Library
            \end{enumerate}
            
            \item[2.2] \textbf{การออกแบบสถาปัตยกรรมซอฟต์แวร์ (13\%)}
            \begin{enumerate}
                \item[2.2.1] ออกแบบ System Architecture และ Component Diagram
                \item[2.2.2] ออกแบบ Database Schema และ Data Model
                \item[2.2.3] ออกแบบ API Architecture และ Endpoints
                \item[2.2.4] ออกแบบ Security Architecture และ Authentication
                \item[2.2.5] ออกแบบ Deployment Architecture และ Infrastructure
                \item[2.2.6] สร้าง Technical Specifications Document
            \end{enumerate}
        \end{enumerate}

        \vspace{2.5cm}

        \textbf{Phase 3: การพัฒนาระบบ (Development Phase)} \textlight{- 40\% ของโครงงานทั้งหมด}

        \begin{enumerate}
            \item[3.1] \textbf{การพัฒนาระบบส่วนแกนหลัก (Core System Development) (40\%)}
            \begin{enumerate}
                \item[3.1.1] \textbf{Backend Development (20\%)}
                \begin{enumerate}
                    \item[3.1.1.1] สร้าง API สำหรับการอัปโหลดและประมวลผล PDF (4\%)
                    \item[3.1.1.2] พัฒนาระบบ Co-occurrence Analysis Engine (6\%)
                    \item[3.1.1.3] สร้างระบบ Graph Generation และ Management (5\%)
                    \item[3.1.1.4] พัฒนาระบบ Database และ Data Storage (3\%)
                    \item[3.1.1.5] บูรณาการกับ Large Language Models (LLM) (2\%)
                \end{enumerate}
                
                \item[3.1.2] \textbf{Frontend Development (15\%)}
                \begin{enumerate}
                    \item[3.1.2.1] สร้าง Dashboard และ Main Interface (4\%)
                    \item[3.1.2.2] พัฒนา File Upload และ Management System (3\%)
                    \item[3.1.2.3] สร้าง Interactive Graph Visualization (5\%)
                    \item[3.1.2.4] พัฒนา Search และ Filter Components (2\%)
                    \item[3.1.2.5] สร้าง Export และ Share Features (1\%)
                \end{enumerate}
                
                \item[3.1.3] \textbf{Integration และ API Development (5\%)}
                \begin{enumerate}
                    \item[3.1.3.1] Integration Testing ระหว่าง Frontend และ Backend (2\%)
                    \item[3.1.3.2] สร้าง RESTful API Documentation (1\%)
                    \item[3.1.3.3] พัฒนา Error Handling และ Logging System (1\%)
                    \item[3.1.3.4] Optimization และ Performance Tuning (1\%)
                \end{enumerate}
            \end{enumerate}
        \end{enumerate}

        \textbf{Phase 4: การทดสอบระบบ (Testing Phase)} \textlight{- 15\% ของโครงงานทั้งหมด}

        \begin{enumerate}
            \item[4.1] \textbf{การทดสอบระบบส่วนแกนหลัก (8\%)}
            \begin{enumerate}
                \item[4.1.1] Unit Testing สำหรับแต่ละ Component (2\%)
                \item[4.1.2] Integration Testing สำหรับระบบทั้งหมด (3\%)
                \item[4.1.3] Performance Testing และ Load Testing (2\%)
                \item[4.1.4] Functional Testing และ User Acceptance Testing (1\%)
            \end{enumerate}
            
            \item[4.2] \textbf{การทดสอบความปลอดภัยของระบบ (4\%)}
            \begin{enumerate}
                \item[4.2.1] Security Testing และ Vulnerability Assessment (2\%)
                \item[4.2.2] Authentication และ Authorization Testing (1\%)
                \item[4.2.3] Data Protection และ Privacy Testing (1\%)
            \end{enumerate}
            
            \item[4.3] \textbf{การทดสอบสภาพแวดล้อมจริง (3\%)}
            \begin{enumerate}
                \item[4.3.1] Deployment Testing บน Production Environment (1\%)
                \item[4.3.2] End-to-End Testing กับข้อมูลจริง (1\%)
                \item[4.3.3] User Experience Testing และ Feedback Collection (1\%)
            \end{enumerate}
        \end{enumerate}

        \vspace{1.6cm}

        \textbf{Phase 5: การจัดทำเอกสารและส่งมอบ (Documentation \& Delivery Phase)} \textlight{- 5\% ของโครงงานทั้งหมด}

        \begin{enumerate}
            \item[5.1] \textbf{การจัดทำเอกสารแพลตฟอร์ม (3\%)}
            \begin{enumerate}
                \item[5.1.1] เขียน User Manual และ Administrator Guide (1\%)
                \item[5.1.2] สร้าง API Documentation และ Technical Guide (1\%)
                \item[5.1.3] จัดทำ Installation Guide และ Troubleshooting (0.5\%)
                \item[5.1.4] สร้าง Video Tutorial และ Demo Materials (0.5\%)
            \end{enumerate}
            
            \item[5.2] \textbf{การส่งมอบแพลตฟอร์ม (2\%)}
            \begin{enumerate}
                \item[5.2.1] Final System Testing และ Quality Assurance (0.5\%)
                \item[5.2.2] Knowledge Transfer และ Training Session (0.5\%)
                \item[5.2.3] Project Handover และ Final Presentation (0.5\%)
                \item[5.2.4] Post-Implementation Support Planning (0.5\%)
            \end{enumerate}
        \end{enumerate}

        \vspace{0.5cm}

        \textbf{การติดตามความก้าวหน้า (Progress Monitoring)}

        \hspace{1cm}การประเมินความก้าวหน้าจะดำเนินการโดยใช้ระบบ Milestone-based Tracking ซึ่งแบ่งการประเมินออกเป็น:

        \begin{enumerate}
            \item[5.3] \textbf{Weekly Progress Review (รายสัปดาห์)}
            \begin{enumerate}
                \item[5.3.1] ตรวจสอบความก้าวหน้าตาม Timeline ที่กำหนด
                \item[5.3.2] ประเมินคุณภาพของงานที่ส่งมอบในแต่ละ Phase
                \item[5.3.3] ระบุและแก้ไข Issues หรือ Blockers ที่เกิดขึ้น
                \item[5.3.4] ปรับปรุง Plan หากจำเป็น
            \end{enumerate}
            
            \item[5.4] \textbf{Phase Gate Reviews (รายเฟส)}
            \begin{enumerate}
                \item[5.4.1] การประเมินผลลัพธ์ที่ได้จากแต่ละ Phase
                \item[5.4.2] การตรวจสอบคุณภาพตาม Acceptance Criteria
                \item[5.4.3] การอนุมัติให้ดำเนินการใน Phase ถัดไป
                \item[5.4.4] การจัดทำ Lessons Learned สำหรับปรับปรุงในอนาคต
            \end{enumerate}
        \end{enumerate}

        \vspace{0.3cm}

        \textbf{สรุปการแบ่งสัดส่วนงาน}
        
        \begin{itemize}
            \item การวางแผนและวิเคราะห์: 15\%
            \item การออกแบบระบบ: 25\%  
            \item การพัฒนาระบบ: 40\%
            \item การทดสอบระบบ: 15\%
            \item การจัดทำเอกสารและส่งมอบ: 5\%
        \end{itemize}

        \hspace{1cm}การแบ่งสัดส่วนดังกล่าวช่วยให้สามารถติดตามความก้าวหน้าได้อย่างแม่นยำ และสามารถประเมินได้ว่าโครงงานดำเนินไปตามแผนที่วางไว้หรือไม่ รวมถึงช่วยในการจัดสรรทรัพยากรและเวลาให้เหมาะสมกับความสำคัญของแต่ละขั้นตอน
    }

    \item[2.7] ทรัพยากรที่ต้องใช้ในการจัดทำโครงงานพิเศษ

\end{enumerate}

\end{document}