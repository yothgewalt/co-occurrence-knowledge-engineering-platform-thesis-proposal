\documentclass[12pt,a4paper]{article}

% Page setup
\usepackage[a4paper,top=1in,bottom=1in,left=1.5in,right=1in]{geometry}
\usepackage{setspace}
\onehalfspacing

% Essential packages
\usepackage{etoolbox}
\usepackage[nonumberlist]{glossaries}
\usepackage{hyperref}
\usepackage{graphicx}
\usepackage{float}

% Table packages
\usepackage{array}
\usepackage{booktabs}
\usepackage{longtable}
\usepackage{tabularx}
\usepackage{multirow}
\usepackage{multicol}

% Thai language support
\usepackage{xunicode}
\usepackage{xltxtra}

% Optimized fonts for Thai language - using Medium weight
\setmainfont[
    Path=./font/Sarabun/,
    UprightFont=*-Regular,
    BoldFont=*-Medium,
    ItalicFont=*-Italic,
    BoldItalicFont=*-MediumItalic,
    Scale=1.0,
    Ligatures=TeX,
    WordSpace=1.2,
    PunctuationSpace=1.1
]{Sarabun}

\newfontfamily\thaifont[
    Path=./font/Sarabun/,
    UprightFont=*-Regular,
    BoldFont=*-Medium,
    ItalicFont=*-Italic,
    BoldItalicFont=*-MediumItalic,
    Scale=1.0,
    Ligatures=TeX,
    WordSpace=1.2,
    PunctuationSpace=1.1,
    Script=Thai,
    Language=Thai
]{Sarabun}

% Light weight font family optimized for Thai
\newfontfamily\thailightfont[
    Path=./font/Sarabun/,
    UprightFont=*-Light,
    BoldFont=*-Medium,
    ItalicFont=*-LightItalic,
    BoldItalicFont=*-MediumItalic,
    Scale=1.0,
    Ligatures=TeX,
    WordSpace=1.2,
    PunctuationSpace=1.1,
    Script=Thai,
    Language=Thai
]{Sarabun}

% Font commands for easy switching
\newcommand{\textlight}[1]{{\thailightfont #1}}
\newcommand{\thailight}{\thailightfont}

% Thai typography optimization commands
\newcommand{\optimalthai}[1]{{\thaifont #1}}

% Additional packages for better Thai support
\usepackage{polyglossia}
\setdefaultlanguage{thai}
\setotherlanguage{english}

% Thai line breaking settings - optimized
\XeTeXlinebreaklocale "th"
\XeTeXlinebreakskip = 0pt plus 2pt minus 1pt

% Thai typography enhancements
\frenchspacing  % Better spacing for Thai text
\tolerance=1000  % Allow more flexibility in line breaking
\emergencystretch=3em  % Emergency stretch for better Thai line breaking

% Thai punctuation spacing
\XeTeXinterchartokenstate = 1

% Better Thai paragraph spacing
\setlength{\parskip}{0.5\baselineskip plus 0.2\baselineskip minus 0.1\baselineskip}
\setlength{\parindent}{2em}

\begin{document}

% Title page
\begin{center}
\hfill\textlight{ทก.01}\\[1cm]

\Large\textbf{แบบเสนอโครงงานพิเศษ (ปริญญานิพนธ์)}\\[0.5cm]
\textbf{สาขาวิชาวิศวกรรมสารสนเทศและเครือข่าย ภาควิชาเทคโนโลยีสารสนเทศ}\\
\textbf{คณะเทคโนโลยีและการจัดการอุตสาหกรรม}
\end{center}

\thispagestyle{empty}
\vspace{2cm}

\section{ข้อมูลขั้นต้นของโครงงาน}

\subsection{ชื่อโครงงาน}
\textbf{(ภาษาไทย)} สมาร์ทฟิต รูทีน\\
\textbf{(ภาษาอังกฤษ)} Smart Fit Routine

\subsection{ชื่อนักศึกษาผู้ทำโครงงาน}
1) นางสาวลลนา สุขรักษ์

\subsection{ชื่ออาจารย์ที่ปรึกษา}
1) อาจารย์สุภาพร ซิ้มเจริญ

\section{รายละเอียดของโครงงาน}

\subsection{ความเป็นมาและความสำคัญของปัญหา}
[เนื้อหาความเป็นมาและความสำคัญของปัญหา...]

\subsection{วัตถุประสงค์ของการจัดทำโครงงานพิเศษ}
\begin{enumerate}
    \item[2.2.1] เพื่อพัฒนาเว็บแอปพลิเคชันที่ช่วยให้ผู้ใช้งานสร้างแผนการออกกำลังกายรายสัปดาห์ได้อย่างง่ายดายและเหมาะสมกับเป้าหมายของตนเอง
    \item[2.2.2] เพื่อให้คำแนะนำท่าออกกำลังกายที่เหมาะสมกับความต้องการของผู้ใช้ โดยอ้างอิงจากฐานข้อมูลที่มีรายละเอียดครบถ้วน
    \item[2.2.3] เพื่อออกแบบอินเตอร์เฟซที่ใช้งานง่ายและสะดวกต่อการเข้าถึงข้อมูลและการจัดการแผนการออกกำลังกายของผู้ใช้
\end{enumerate}

\subsection{ขอบเขตของการทำโครงงานพิเศษ (Scope of Special Project)}
\begin{enumerate}
    \item[2.3.1] การพัฒนาเว็บแอพพลิเคชันสมาร์ทฟิต รูทีน ผู้ใช้สามารถลงทะเบียนเพื่อสร้างบัญชีผู้ใช้ใหม่ โดยกรอกข้อมูลส่วนตัว
    \begin{enumerate}
        \item[2.3.1.1] ชื่อ-สกุล
        \item[2.3.1.2] อีเมล
        \item[2.3.1.3] ชื่อผู้ใช้ (Username)
        \item[2.3.1.4] รหัสผ่าน (Password)
    \end{enumerate}
    หลังจากลงทะเบียน ผู้ใช้งานสามารถลงชื่อเข้าใช้งานระบบและยังสามารถเข้าถึงฟีเจอร์ต่าง ๆ ภายในเว็บแอพพลิเคชันได้
    
    \item[2.3.2] การสร้างและจัดการกับแผนการการออกกำลังกายจะถูกแบ่งออกเป็น 2 ข้อ ได้แก่
    \begin{enumerate}
        \item[2.3.2.1] ผู้ใช้สามารถสร้างแผนการออกกำลังกายส่วนตัวโดยเลือกท่าทางการออกกำลังกายจากฐานข้อมูล ซึ่งแต่ละท่าจะมีรายละเอียด ได้แก่
        \begin{enumerate}
            \item[2.3.2.1.1] ชื่อท่า
            \item[2.3.2.1.2] คำอธิบาย
            \item[2.3.2.1.3] วิดีโอสาธิต
        \end{enumerate}
        \item[2.3.2.2] ผู้ใช้สามารถบันทึกแผนการออกกำลังกายที่ตนเองสร้างขึ้นไว้ในระบบ และสามารถเรียกดูหรือปรับแก้แผนได้ตามต้องการ
    \end{enumerate}
    
    \item[2.3.3] ระบบสามารถแนะนำการออกกำลังกายที่เข้ากับเป้าหมายของผู้ใช้งานได้
    \item[2.3.4] เว็บแอพพลิเคชันจะมีการแจ้งเตือนผู้ใช้งานเกี่ยวกับการออกกำลังกาย หรือแจ้งเตือนเกี่ยวกับคำแนะนำใหม่ ๆ ที่เหมาะสมกับผู้ใช้
    \item[2.3.5] ฐานข้อมูลจะถูกจัดเตรียมเพื่อจัดเก็บข้อมูลสำคัญ ได้แก่
    \begin{enumerate}
        \item[2.3.5.1] ข้อมูลผู้ใช้ (User Information)
        \item[2.3.5.2] ข้อมูลท่าการออกกำลังกาย (Exercise Details)
        \item[2.3.5.3] ข้อมูลการติดตามความก้าวหน้าของผู้ใช้ (Progress Tracking)
    \end{enumerate}
\end{enumerate}

\subsection{รายละเอียดทฤษฎีที่ใช้ในการจัดทำปริญญานิพนธ์}

\subsubsection{ทฤษฎีที่เกี่ยวกับ Web Application}

\paragraph{1. Client-Server Architectures} เป็นรูปแบบของการออกแบบระบบคอมพิวเตอร์ที่แบ่งหน้าที่และภาระงานระหว่างคอมพิวเตอร์ที่เรียกว่าไคลเอนท์ (Clients) และคอมพิวเตอร์ที่เรียกว่าเซิร์ฟเวอร์ (Servers) เพื่อให้การทำงานและการประมวลผลเกิดขึ้นอย่างมีประสิทธิภาพและมีความสมดุล สิ่งที่ทำให้แบบไคลเอนท์เซิร์ฟเวอร์มีประโยชน์หลัก 4 ประการได้แก่:

\begin{enumerate}
    \item[1.1] สามารถปรับเพิ่มลดได้ (Scalable)
    \item[1.2] สนับสนุนความหลากหลายรูปแบบ
    \item[1.3] การแยกแยะและการปรับปรุงง่าย
    \item[1.4] ความเสถียรและความยืดหยุ่นในการเปลี่ยนแปลง
\end{enumerate}

\paragraph{2. HTTP (Hypertext Transfer Protocol)} คือโปรโตคอลการสื่อสารระหว่างคอมพิวเตอร์ที่ใช้กันบนเว็บไซต์และบนอินเทอร์เน็ตเพื่อแลกเปลี่ยนข้อมูลและแสดงผลในรูปแบบของเอกสาร...

\paragraph{3. REST (Representational State Transfer)} เป็นแนวคิดทางสถาปัตยกรรมในการสื่อสารระหว่างระบบคอมพิวเตอร์บนเว็บ...

\paragraph{4. การใช้ HTTP Verbs ในการกระทำ:} REST ใช้เมธอด (HTTP Verbs) เพื่อให้กำหนดการกระทำต่าง ๆ กับทรัพยากร เช่น
\begin{enumerate}
    \item[4.1] GET: ดึงข้อมูลทรัพยากรหรือคอลเลกชันของทรัพยากร
    \item[4.2] POST: สร้างทรัพยากรใหม่
    \item[4.3] PUT: อัปเดตข้อมูลทรัพยากรเฉพาะ
    \item[4.4] DELETE: ลบทรัพยากรที่ระบุ
\end{enumerate}

\subsubsection{รายงานการค้นคว้า การศึกษา หรือการวิจัยที่เกี่ยวข้อง}
**** สอง - สาม งาน (กำลังดูและเลือกรายงานที่เหมาะสมกับงานมากที่สุด)

\subsection{วิธีการดำเนินงานจัดทำโครงงานพิเศษ}
ภาคการศึกษาที่ 1/2566\\
ภาคการศึกษาที่ 2/2566

\subsection{แผนกิจกรรมและตารางเวลาในการจัดทำ}

\subsubsection{แผนกิจกรรมหลักและระยะเวลา}

\paragraph{ภาคการศึกษาที่ 1}
\begin{table}[h!]
    \centering
    \caption{แผนการดำเนินงานภาคการศึกษาที่ 1}
    \begin{tabular}{|l|c|c|c|c|c|c|c|c|c|c|c|c|c|c|c|c|}
        \hline
        \multirow{2}{*}{\textbf{ขั้นตอนการดำเนินงาน}} & \multicolumn{4}{c|}{\textbf{กรกฎาคม}} & \multicolumn{4}{c|}{\textbf{สิงหาคม}} & \multicolumn{4}{c|}{\textbf{กันยายน}} & \multicolumn{4}{c|}{\textbf{ตุลาคม}} \\
        \cline{2-17}
        & 1 & 2 & 3 & 4 & 1 & 2 & 3 & 4 & 1 & 2 & 3 & 4 & 1 & 2 & 3 & 4 \\
        \hline
        1 & & & & & & & & & & & & & & & & \\
        \hline
        2 & & & & & & & & & & & & & & & & \\
        \hline
        3 & & & & & & & & & & & & & & & & \\
        \hline
        4 & & & & & & & & & & & & & & & & \\
        \hline
        5 & & & & & & & & & & & & & & & & \\
        \hline
    \end{tabular}
    \label{tab:semester1}
\end{table}

\paragraph{ภาคการศึกษาที่ 2}
\begin{table}[h!]
    \centering
    \caption{แผนการดำเนินงานภาคการศึกษาที่ 2}
    \begin{tabular}{|l|c|c|c|c|c|c|c|c|c|c|c|c|c|c|c|c|}
        \hline
        \multirow{2}{*}{\textbf{ขั้นตอนการดำเนินงาน}} & \multicolumn{4}{c|}{\textbf{ธันวาคม}} & \multicolumn{4}{c|}{\textbf{มกราคม}} & \multicolumn{4}{c|}{\textbf{กุมภาพันธ์}} & \multicolumn{4}{c|}{\textbf{มีนาคม}} \\
        \cline{2-17}
        & 1 & 2 & 3 & 4 & 1 & 2 & 3 & 4 & 1 & 2 & 3 & 4 & 1 & 2 & 3 & 4 \\
        \hline
        1 & & & & & & & & & & & & & & & & \\
        \hline
        2 & & & & & & & & & & & & & & & & \\
        \hline
        3 & & & & & & & & & & & & & & & & \\
        \hline
        4 & & & & & & & & & & & & & & & & \\
        \hline
        5 & & & & & & & & & & & & & & & & \\
        \hline
        6 & & & & & & & & & & & & & & & & \\
        \hline
        7 & & & & & & & & & & & & & & & & \\
        \hline
    \end{tabular}
    \label{tab:semester2}
\end{table}

\subsection{ทรัพยากรที่ต้องใช้ในการจัดทำโครงงานพิเศษ}

\subsubsection{เครื่องมือในการจัดทำโครงงานพิเศษ}

\paragraph{Software}
\begin{enumerate}
    \item[2.7.1.1.1] MySQL
    \item[2.7.1.1.2] NLP
    \item[2.7.1.1.3] Python
    \item[2.7.1.1.4] Power BI
\end{enumerate}

\paragraph{Hardware}
\begin{enumerate}
    \item[2.7.1.2.1] คอมพิวเตอร์/โน๊ตบุ๊ค
    \item[2.7.1.2.2] โทรศัพท์
\end{enumerate}

\subsubsection{งบประมาณที่ใช้ในการจัดทำ}
\begin{itemize}
    \item ค่าจัดทำปริญญานิพนธ์ 1,000 บาท
    \item ค่าใช้จ่ายอื่น ๆ 500 บาท
    \item รวมเป็นเงิน 1,500 บาท
\end{itemize}

\subsection{ผลที่คาดว่าจะได้รับ (ปรับตามวัตถุประสงค์)}
\begin{enumerate}
    \item[2.8.1] ผู้ใช้สามารถหาเมนูอาหารที่ตรงตามความต้องการได้
    \item[2.8.2] Chatbot สามารถที่จะตอบโต้และแนะนำเมนูอาหารได้
    \item[2.8.3] ผู้ใช้ได้รับคำแนะนำอย่างเหมาะสมในด้านการรับประทานเมนูอาหาร
\end{enumerate}

\subsection{เอกสารอ้างอิง}
** ใช้ Zotero เลือกเป็น APA7th

\subsection{ภาคผนวก}
** ใช้ figma หรือ wireframe

\vspace{3cm}

% Signature section
\begin{minipage}{0.5\textwidth}
ลงชื่อ..................................ผู้เสนอโครงงาน\\
(นายพรเทพ )
\end{minipage}
\begin{minipage}{0.5\textwidth}
ลงชื่อ..................................ผู้เสนอโครงงาน\\
(นางสาวลลนา สุขรักษ์)
\end{minipage}

\vspace{1cm}
วันที่ยื่นเสนอโครงงาน........../............./............

\vspace{2cm}

\section*{ความเห็นอาจารย์ที่ปรึกษาโครงงาน}
\dotfill\\
\dotfill\\
\dotfill\\

\vspace{1cm}
ลงชื่อ..................................อาจารย์ที่ปรึกษา\\
(……….…………………….……….)\\
วันที่................/..................../.................

\vspace{1cm}
สาขาวิชา / ภาควิชาที่ได้รับแบบเสนอโครงงานวันที่ ..................................................................................

\vspace{1cm}

\section*{ผลการพิจารณา}
\dotfill\\
\dotfill\\
\dotfill\\

\vspace{1cm}

\begin{minipage}{0.5\textwidth}
ลงชื่อ..................................ประธาน\\
(……….…………………….……….)\\
วันที่................/..................../.................

\vspace{1cm}

ลงชื่อ..................................กรรมการ\\
(……….…………………….……….)\\
วันที่................/..................../.................
\end{minipage}
\begin{minipage}{0.5\textwidth}
ลงชื่อ..................................กรรมการ\\
(……….…………………….……….)\\
วันที่................/..................../.................

\vspace{1cm}

ลงชื่อ..................................กรรมการ\\
(……….…………………….……….)\\
วันที่................/..................../................
\end{minipage}

\end{document}