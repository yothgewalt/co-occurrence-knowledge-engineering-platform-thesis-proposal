\documentclass{article}

\usepackage{etoolbox}
\usepackage[a4paper,top=1in,bottom=1in,left=1in,right=1in]{geometry}
\usepackage[nonumberlist]{glossaries}

% Thai language support
\usepackage{fontspec}
\usepackage{xunicode}
\usepackage{xltxtra}

% Set fonts for Thai language
\setmainfont[Path=./font/Sarabun/,
    UprightFont=*-Regular,
    BoldFont=*-Bold,
    ItalicFont=*-Italic,
    BoldItalicFont=*-BoldItalic]{Sarabun}

\newfontfamily\thaifont[Path=./font/Sarabun/,
    UprightFont=*-Regular,
    BoldFont=*-Bold,
    ItalicFont=*-Italic,
    BoldItalicFont=*-BoldItalic]{Sarabun}

% Additional packages for better Thai support
\usepackage{polyglossia}
\setdefaultlanguage{thai}
\setotherlanguage{english}

% Thai line breaking settings
\XeTeXlinebreaklocale "th"
\XeTeXlinebreakskip = 0pt plus 2pt minus 1pt

\title{ข้อเสนอโครงการ}
\author{ชื่อผู้เขียน}
\date{\today}

\begin{document}

\maketitle

\section{บทนำ}
นี่คือตัวอย่างเอกสาร LaTeX ที่รองรับภาษาไทย โดยใช้ฟอนต์ Sarabun

\section{วัตถุประสงค์}
\begin{itemize}
    \item วัตถุประสงค์ที่ 1
    \item วัตถุประสงค์ที่ 2
    \item วัตถุประสงค์ที่ 3
\end{itemize}

\section{Introduction (English)}
\begin{english}
This section demonstrates bilingual support. The document can contain both Thai and English text.
\end{english}

\section{สรุป}
เอกสารนี้แสดงให้เห็นการใช้ภาษาไทยใน LaTeX อย่างสมบูรณ์

\end{document}

